\documentclass{article}
\usepackage{amsmath}

\title{Modelos de toma de decisión heurística para el problema del bar El Farol}
\author{Semillero de modelación y simulación de fenómenos sociales}
\date{}

\newcommand{\attrac}{\mbox{Attract}}
\newcommand{\logi}{\mbox{logistic}}
\newcommand{\dist}{\mbox{dist}}
\newcommand{\fradist}{\mbox{\textsc{FRAdist}}}
\newcommand{\bias}[1]{\ensuremath{\mbox{\textsc{bias}$_{\mbox{\tiny ${#1}$}}$}}}

\begin{document}

\subsubsection*{Toma de decisiones}

Los modelos descritos a continuación basan el proceso de toma de decisiones en una elección aleatoria entre las estrategias focales. La probabilidad de elegir una estrategia focal $k$, $P(k)$, está dada por la fórmula
%
\begin{equation}\label{eq:probabilidades}
P(k)=\frac{\attrac(k, i, s)}{\sum_{k'}\attrac(k',i,s)}
\end{equation}
%
en donde $i$ es el historial de asistencia del jugador y $s$ el puntaje obtenido en la última ronda. La diferencia entre modelos consistirá en la manera en que definen la función $\attrac$.

\subsubsection*{Modelo MBiases}

Este modelo sólo considera los sesgos a priori de cada una de las regiones focales. De esta manera,
%
\begin{equation}\label{eq:MBiases}
\attrac(k, i, s)=\bias{k}
\end{equation}
%
Para la región \textsc{rs}, se tiene 
%
\begin{equation}\label{eq:MBiases}
\attrac(\mbox{\textsc{rs}}, i, s)=\bias{\mbox{\textsc{rs}}} =  1-\sum_{k}\bias{k}
\end{equation}
%

\subsubsection*{Modelo WSLS}

Este modelo considera, además de los sesgos a priori, la heurística win stay, lose shift. De esta manera,
%
\begin{equation}\label{eq:WSLS}
\attrac(k, i, s)=\bias{k} + \alpha*\logi(s,\beta,\gamma)*I(k,i)
\end{equation}
%
donde
%
\[
I(k,i)=\begin{cases}
1,&\mbox{ si }k=i\\
0,&\mbox{ en otro caso}\\
\end{cases}
\]

\subsubsection*{Modelo FRA}

Finalmente, el modelo \textsc{fra} considera la heurística basada en estrategias focales como atractores. Se tiene,
%
\begin{align}\label{eq:FRA}
\attrac(k, i, s, j)=&\bias{k} + \alpha*\logi(s,\beta,\gamma)*I(k,i)  \\
&+ 1-\delta*\logi(\fradist(k,i),\epsilon,\zeta) \nonumber
\end{align}
%
donde
%
\[
\fradist(k,i)=\dist(k,i) + \dist(k^c, j)\qquad  \dist(k,i)=\sum_{n=1}^{NumLoc}|k_n-i_n|
\]
%
y $k^c$ es el complemento de $k$ y $j$ es el vector de `overcrowding', donde la componente $j_n$ es 1 si la asistencia la bar en la ronda $n$ superó el umbral y 0 si no.  


\end{document}